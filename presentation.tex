%% LaTeX Beamer presentation template (requires beamer package)
%% see http://bitbucket.org/rivanvx/beamer/wiki/Home
%% idea contributed by H. Turgut Uyar
%% template based on a template by Till Tantau
%% this template is still evolving - it might differ in future releases!

\documentclass{beamer}

\mode<presentation>
{
\usetheme{Warsaw}

\setbeamercovered{transparent}
}

\usepackage[english]{babel}
\usepackage[utf8]{inputenc}

% font definitions, try \usepackage{ae} instead of the following
% three lines if you don't like this look
\usepackage{mathptmx}
\usepackage[scaled=.90]{helvet}
\usepackage{courier}
%\usepackage{ae}


\usepackage[T1]{fontenc}

\usepackage{multicol}

\usepackage[round]{natbib}


\title{Acessibilidade nas fases de Engenharia de Requisitos, Projeto e Codificação de Software: Uma ferramenta de apoio}

\subtitle{Defesa de Dissertação de Mestrado}

% - Use the \inst{?} command only if the authors have different
%   affiliation.
%\author{F.~Author\inst{1} \and S.~Another\inst{2}}
\author{\mbox{Rodrigo G. de ~Branco} \and \mbox{Profª. Drª. Débora M. B. ~Paiva (Orientadora)}}

% - Use the \inst command only if there are several affiliations.
% - Keep it simple, no one is interested in your street address.
\institute[Universities of]
{
Faculdade de Computação\\
Universidade Federal de Mato Grosso do Sul
}

\date{09 de Setembro de 2013}


% This is only inserted into the PDF information catalog. Can be left
% out.
\subject{Talks}



% If you have a file called "university-logo-filename.xxx", where xxx
% is a graphic format that can be processed by latex or pdflatex,
% resp., then you can add a logo as follows:

\pgfdeclareimage[height=0.5cm]{university-logo}{university-logo-filename}
\logo{\pgfuseimage{university-logo}}



% Delete this, if you do not want the table of contents to pop up at
% the beginning of each subsection:
\AtBeginSubsection[]
{
\begin{frame}<beamer>
\frametitle{Roteiro}
	\begin{multicols}{2}
		\tableofcontents[currentsection,currentsubsection]
	\end{multicols}
\end{frame}
}

% If you wish to uncover everything in a step-wise fashion, uncomment
% the following command:

%\beamerdefaultoverlayspecification{<+->}

\begin{document}

\begin{frame}
\titlepage
\end{frame}

\begin{frame}
\frametitle{Roteiro}
	\begin{multicols}{2}
		\tableofcontents
	\end{multicols}
% You might wish to add the option [pausesections]
\end{frame}


\section{Introdução e Motivação} 

\subsection[Motivação]{Motivação}

\begin{frame}
\frametitle{Internet}
\framesubtitle{}

\begin{itemize}
  \item Importante meio de disseminação de informação
  \item Ferramenta essencial nas atividades cotidianas
  \item Alguns serviços disponibilizados \textbf{APENAS} por essa via \citep{irpf:13,tjce:11}
\end{itemize}

\end{frame}

\begin{frame}
\frametitle{Utilização da Internet como apoio aos negócios}
\framesubtitle{Benefícios \citep{oliveira:11}}

\begin{itemize}
  \item Disponibilidade de 24 horas por dia;
  \item Possibilidade de acesso de todas as partes do planeta - ou fora dele? \citep{curiosity:13};
  \item Necessidade de espaço físico e de infra-estrutura reduzidos (ex: bancos) para realizar;
as atividades;
  \item Custo de investimento inicial baixo, entre outros.
\end{itemize}

\end{frame}

\begin{frame}
\frametitle{Acesso equalitário à informação}
\framesubtitle{}

\begin{itemize}
  \item Acesso homogêneo aos recursos da Internet \citep{5260918}
  \item Única forma de acesso à informação por alguns grupos de usuários
  \item Pessoas diferentes com problemas diferentes, equipamentos diferentes, \textit{softwares} diferentes e necessidades diferentes
  \item Pessoas com deficiências são prejudicadas por soluções malfeitas
\end{itemize}

\end{frame}

\begin{frame}
\frametitle{Produto \textit{web} acessível}
\framesubtitle{}

\begin{itemize}
  \item É difícil garantir um produto 100\% acessível
  \item Domínio de estudo relativamente novo
  \item Várias pesquisas na área \citep{lazar:04,brajnik:06,zeng:05}
  \item Papel fundamental da Engenharia de \textit{Software} no processo de desenvolvimento
  \item Os custos são menores quando a acessibilidade é considerada durante o processo de desenvolvimento
do \textit{software} \citep{groves:11}
\end{itemize}

\end{frame}

\subsection[Problema]{Problema}

\begin{frame}[allowframebreaks]
\frametitle{Observações}
\framesubtitle{}

\begin{itemize}
  \item Integração de tópicos de Acessibilidade no processo de desenvolvimento \citep{springerlink:10.1007/978-3-642-02713-0,maia:10}
  \item Rastreabilidade dos requisitos de acessibilidade
  \item Muitos desenvolvedores não sabem como codificar de forma a tornar seus produtos acessíveis \citep{1630123,alves:11}
  \item Desenvolvedores não estão satisfeitos com as ferramentas de apoio à acessibilidade disponíveis \citep{Trewin:2010:ACT:1805986.1806029}
  \item Desenvolvedores estão insatisfeitos em utilizar ferramentas externas ao
seu ambiente de desenvolvimento para efetuar a avaliação \citep{Trewin:2010:ACT:1805986.1806029}
  \item As ferramentas nem sempre informam de forma objetiva as mudanças necessárias para fornecer um produto acessível \citep{groves:12}
  \item A avaliação normalmente ocorre quando o produto já está pronto (Refatoramento)  
\end{itemize}

\end{frame}

\subsection[Objetivos]{Objetivos}

\begin{frame}
\frametitle{Objetivos gerais}
\framesubtitle{}

\begin{itemize}
 \item Estender o MTA, propondo uma metologia para a rastreabilidade dos requisitos de acessibilidade atravées do processo de desenvolvimento de \textit{software}
 \item Permitir a associação explícita entre os requisitos de acessibilidade e os artefatos de documentação e, para cada associação, especificar uma ou mais técnicas de implementação de acessibilidade de acordo com o documento de conformidade em acessibilidade escolhido
 \item Implementar uma ferramenta de suporte que seja integrada ao ambiente de desenvolvimento e que implemente os objetivos listados acima
\end{itemize}

\end{frame}

\begin{frame}[allowframebreaks]
\frametitle{Características desejáveis para a ferramenta}
\framesubtitle{Partindo da pesquisa de \citet{Trewin:2010:ACT:1805986.1806029}}

\begin{itemize}
 \item Seja orientada ao desenvolvedor (a apresentação dos resultados nas ferramentas tradicionais são adequadas para avaliação e auditoria de sites, e não para desenvolvedores)
 \item Seja integrada ao ambiente de desenvolvimento do desenvolvedor
 \item Apresente informações objetivas e no momento em que o desenvolvedor desejar visualizar
 \item Tenha relação direta entre os requisitos e casos de uso com a etapa de codificação
 \item Permita que seja feita o rastreamento dos requisitos de acessibilidade, desde a sua concepção até as fases de codicação
 \item Permita que o desenvolvedor consiga verificar, em nível de código, a associação dos requisitos e modelos
\end{itemize}

\end{frame}

\subsection[Metodologia]{Metodologia}

\begin{frame}[allowframebreaks]
\frametitle{Passos necessários para atingir aos objetivos}
\framesubtitle{}

\begin{itemize}
 \item Estudar a literatura sobre o assunto
 \item Identificar os pontos de integração entre as atividades de engenharia de requisitos, projeto e geração de cóodigo
 \item Estudar o processo de desenvolvimento de plugins para o Eclipse
 \item Estudar como técnicas de acessibilidade podem ser associadas aos modelos
 \item Estudar quais tecnologias existentes podem ser usadas para efetuar a associação dos requisitos e modelos às téecnicas de acessibilidade
 \item Desenvolver a ferramenta
 \item Efetuar uma prova de conceito, criando um projeto utilizando o MTA e a ferramenta proposta
\end{itemize}

\end{frame}

\section{Fundamentação Teórica}
 
\subsection[Acessibilidade na Web]{Acessibilidade na Web}

\begin{frame}
\frametitle{Contextualização}
\framesubtitle{}

\begin{itemize}
 \item A Internet foi projetada para para ser usada sem \textit{mouse}, até sem os olhos \citep{thatcher:06}
 \item Tecnologias como \textit{Javascript} e \textit{Flash} deixaram a Internet mais atrativa, mas\ldots
 \pause
 \item o uso indiscriminado dessas tecnologias podem, ao mesmo tempo, facilitar, inibir ou impedir o acesso aos recursos
 \item Desafio: Conscientização dos desenvolvedores \citep{freire:08,alves:11}
\end{itemize}

\end{frame}

\begin{frame}
\frametitle{Tecnologias Assistivas}
\framesubtitle{}

\begin{itemize}
  \item Conjunto de equipamentos, serviços, estratégias e práticas concebidas para atenuar os problemas encontrados pelas pessoas com necessidades especiais \citep{cook:95}
  \item Cegueira
  \item Baixa visão
  \item Deficiência física
  \item Deficiência auditiva
\end{itemize}

\end{frame}

\begin{frame}
\frametitle{Legislação}
\framesubtitle{}

\begin{itemize}
  \item Começou com a \textit{Section 508}, em 1998 \citep{section508:98}
  
\end{itemize}

\end{frame}

\subsection[Pesquisa Bibliográfica]{Pesquisa Bibliográfica}

\subsection[MTA]{Acessibilidade no Processo de Desenvolvimento}

\section{Desenvolvimento}

\subsection[Escopo - MTA]{Escopo - MTA}

\subsection[Ferramentas e Tecnologias]{Ferramentas e Tecnologias}

\subsection[Construção da Ferramenta]{Construção da Ferramenta}

\section{Prova de Conceito}

\subsection[Definição do Projeto]{Definição do Projeto}

\subsection[Modelagem do Sistema]{Modelagem do Sistema}

\subsection[Limitações]{Limitações}

\section{Conclusões}

\subsection[Contribuições]{Contribuições}

\subsection[Trabalhos Futuros]{Trabalhos Futuros}

\section{Referências Bibliográficas}

\begin{frame}[allowframebreaks]

\bibliographystyle{apalike}

\bibliography{sbc-template}

\end{frame}

\end{document}

% \begin{frame}
% \frametitle{}
% \framesubtitle{Subtitles are optional}
% 
% xx
% \begin{itemize}
%   \item
%   \item
% \end{itemize}
% \end{frame}
% 
% \begin{frame}
% \frametitle{}
% 
% % You can create overlays
% \begin{itemize}
%   \item using the \texttt{pause} command:
%   \begin{itemize}
%     \item First item.
%     \pause
%     \item Second item.
%   \end{itemize}
%   \item using overlay specifications:
%   \begin{itemize}
%     \item<3-> First item.
%     \item<4-> Second item.
%   \end{itemize}
%   \item using the general \texttt{uncover} command:
%   \begin{itemize}
%     \uncover<5->{\item First item.}
%     \uncover<6->{\item Second item.}
%   \end{itemize}
% \end{itemize}
% \end{frame}
% 
% \section*{Summary}
% 
% \begin{frame}
% \frametitle<presentation>{Summary}
% 
% \begin{itemize}
%   \item The \alert{first main message} of your talk in one or two lines.
% \end{itemize}
% 
% % The following outlook is optional.
% \vskip0pt plus.5fill
% \begin{itemize}
%   \item Outlook
%   \begin{itemize}
%     \item Something you haven't solved.
%     \item Something else you haven't solved.
%   \end{itemize}
% \end{itemize}
% \end{frame}
% 
% \end{document}
